\documentclass[11pt, a4paper, onecolumn]{article}

% ---------- Packages ----------
% Layout & formatting
\usepackage[margin=0.9in]{geometry}
\usepackage{setspace}
\setstretch{1.15}
\usepackage[skip=6pt plus2pt]{parskip}
\usepackage{titlesec}

% Compact section headings
\titlespacing*{\section}{0pt}{1.2ex plus 0.4ex minus 0.2ex}{0.6ex plus 0.2ex}
\titlespacing*{\subsection}{0pt}{1.0ex plus 0.3ex minus 0.2ex}{0.4ex plus 0.1ex}
\titlespacing*{\subsubsection}{0pt}{0.8ex plus 0.2ex minus 0.1ex}{0.3ex plus 0.1ex}

% Math
\usepackage{amsmath, amssymb, amsthm}

% Graphics
\usepackage{graphicx}
\usepackage{subcaption}
\usepackage{float}

% Tables
\usepackage{booktabs}
\usepackage{multirow}
\usepackage{array}

% Algorithms / pseudocode
\usepackage{algorithm}
\usepackage{algpseudocode}

% Compact captions
\usepackage[font=small, labelfont=bf, skip=4pt]{caption}

% Compact lists
\usepackage{enumitem}
\setlist{nosep, leftmargin=*}

% References & links
\usepackage[hidelinks]{hyperref}
\usepackage{cleveref}
\usepackage[numbers]{natbib}

% Misc
\usepackage{lipsum} % placeholder text, remove in final version

% ---------- Theorem environments ----------
\newtheorem{theorem}{Theorem}[section]
\newtheorem{lemma}[theorem]{Lemma}
\newtheorem{definition}[theorem]{Definition}

% ---------- Title ----------
\title{Report Title}
\author{Author Name \\ \textit{Affiliation} \\ \texttt{email@example.com}}
\date{\today}

% ==========================================================
\begin{document}

\maketitle

% ============================================================
%  SECTION 1 — Equations
% ============================================================
\section{Equations}\label{sec:equations}

An inline equation example: the energy--mass relation is $E = mc^{2}$.

A standalone numbered equation:
\begin{equation}\label{eq:quadratic}
  x = \frac{-b \pm \sqrt{b^{2} - 4ac}}{2a}
\end{equation}

A system of aligned equations:
\begin{align}
  \nabla \cdot  \mathbf{E} &= \frac{\rho}{\varepsilon_0}       \label{eq:gauss} \\
  \nabla \times \mathbf{E} &= -\frac{\partial \mathbf{B}}{\partial t} \label{eq:faraday}
\end{align}

A piecewise function:
\begin{equation}\label{eq:piecewise}
  f(x) =
  \begin{cases}
    x^{2},  & \text{if } x \geq 0, \\
    -x^{2}, & \text{if } x < 0.
  \end{cases}
\end{equation}

A matrix:
\begin{equation}\label{eq:matrix}
  \mathbf{A} =
  \begin{bmatrix}
    a_{11} & a_{12} \\
    a_{21} & a_{22}
  \end{bmatrix}
\end{equation}

\begin{definition}[Convexity]\label{def:convex}
  A function $f\colon \mathbb{R}^n \to \mathbb{R}$ is \emph{convex} if
  $f(\lambda \mathbf{x} + (1-\lambda)\mathbf{y})
   \leq \lambda f(\mathbf{x}) + (1-\lambda) f(\mathbf{y})$
  for all $\mathbf{x},\mathbf{y}$ and $\lambda \in [0,1]$.
\end{definition}

Cross-referencing: \Cref{eq:quadratic} is the quadratic formula;
\Cref{eq:gauss,eq:faraday} are Maxwell's equations.
See Knuth~\cite{knuth1984} for typesetting conventions.

% ============================================================
%  SECTION 2 — Figures
% ============================================================
\section{Figures}\label{sec:figures}

\subsection{Single Figure}
\begin{figure}[htbp]
  \centering
  \includegraphics[width=0.65\textwidth]{example-image}
  \caption{A single figure occupying 65\% of the text width.}
  \label{fig:single}
\end{figure}

\subsection{Two Figures Side by Side}
\begin{figure}[htbp]
  \centering
  \begin{subfigure}[b]{0.48\textwidth}
    \centering
    \includegraphics[width=\textwidth]{example-image-a}
    \caption{Left subfigure.}
    \label{fig:sub_a}
  \end{subfigure}
  \hfill
  \begin{subfigure}[b]{0.48\textwidth}
    \centering
    \includegraphics[width=\textwidth]{example-image-b}
    \caption{Right subfigure.}
    \label{fig:sub_b}
  \end{subfigure}
  \caption{Two subfigures displayed side by side.}
  \label{fig:two_side}
\end{figure}

\subsection{2$\times$2 Grid of Figures}
\begin{figure}[htbp]
  \centering
  \begin{subfigure}[b]{0.48\textwidth}
    \centering
    \includegraphics[width=\textwidth]{example-image-a}
    \caption{Top left.}\label{fig:grid_tl}
  \end{subfigure}\hfill
  \begin{subfigure}[b]{0.48\textwidth}
    \centering
    \includegraphics[width=\textwidth]{example-image-b}
    \caption{Top right.}\label{fig:grid_tr}
  \end{subfigure}

  \vspace{0.5em}

  \begin{subfigure}[b]{0.48\textwidth}
    \centering
    \includegraphics[width=\textwidth]{example-image-a}
    \caption{Bottom left.}\label{fig:grid_bl}
  \end{subfigure}\hfill
  \begin{subfigure}[b]{0.48\textwidth}
    \centering
    \includegraphics[width=\textwidth]{example-image-b}
    \caption{Bottom right.}\label{fig:grid_br}
  \end{subfigure}
  \caption{A 2$\times$2 grid of subfigures.}
  \label{fig:grid}
\end{figure}

\subsection{Three Figures in a Row}
\begin{figure}[htbp]
  \centering
  \begin{subfigure}[b]{0.31\textwidth}
    \centering
    \includegraphics[width=\textwidth]{example-image-a}
    \caption{First.}\label{fig:tri_a}
  \end{subfigure}\hfill
  \begin{subfigure}[b]{0.31\textwidth}
    \centering
    \includegraphics[width=\textwidth]{example-image-b}
    \caption{Second.}\label{fig:tri_b}
  \end{subfigure}\hfill
  \begin{subfigure}[b]{0.31\textwidth}
    \centering
    \includegraphics[width=\textwidth]{example-image-c}
    \caption{Third.}\label{fig:tri_c}
  \end{subfigure}
  \caption{Three subfigures in a single row.}
  \label{fig:triple}
\end{figure}

Cross-referencing: see \Cref{fig:single} for the single figure and
\Cref{fig:sub_a,fig:sub_b} for the pair.
Residual networks~\cite{he2016deep} are often visualised this way.

% ============================================================
%  SECTION 3 — Tables
% ============================================================
\section{Tables}\label{sec:tables}

\subsection{Simple Table}
\begin{table}[htbp]
  \centering
  \caption{Baseline comparison using \texttt{booktabs}.}
  \label{tab:simple}
  \begin{tabular}{lcc}
    \toprule
    Method   & Accuracy (\%) & Runtime (s) \\
    \midrule
    Baseline & 72.3          & 1.20        \\
    Model A  & 85.1          & 2.45        \\
    Model B  & \textbf{91.2} & 4.05        \\
    \bottomrule
  \end{tabular}
\end{table}

\subsection{Table with Multirow and Multicolumn}
\begin{table}[htbp]
  \centering
  \caption{Results grouped by dataset.}
  \label{tab:multi}
  \begin{tabular}{ll cc}
    \toprule
    & & \multicolumn{2}{c}{\textbf{Metrics}} \\
    \cmidrule(lr){3-4}
    \textbf{Dataset} & \textbf{Method} & Precision & Recall \\
    \midrule
    \multirow{2}{*}{CIFAR-10}
      & ResNet-18  & 0.92 & 0.90 \\
      & VGG-16     & 0.89 & 0.87 \\
    \midrule
    \multirow{2}{*}{ImageNet}
      & ResNet-50    & 0.81 & 0.79 \\
      & EfficientNet & 0.84 & 0.82 \\
    \bottomrule
  \end{tabular}
\end{table}

Cross-referencing: \Cref{tab:simple} summarises baselines;
\Cref{tab:multi} groups results by dataset.

% ============================================================
%  SECTION 4 — Algorithms
% ============================================================
\section{Algorithms}\label{sec:algorithms}

\subsection{Simple Algorithm}
\begin{algorithm}[htbp]
  \caption{Binary Search}\label{alg:binary_search}
  \begin{algorithmic}[1]
    \Require Sorted array $A[1 \ldots n]$, target $t$
    \Ensure Index $i$ with $A[i]=t$, or $-1$
    \State $\ell \gets 1,\; r \gets n$
    \While{$\ell \leq r$}
      \State $m \gets \lfloor (\ell + r)/2 \rfloor$
      \If{$A[m] = t$} \Return $m$
      \ElsIf{$A[m] < t$} $\ell \gets m+1$
      \Else{} $r \gets m-1$
      \EndIf
    \EndWhile
    \State \Return $-1$
  \end{algorithmic}
\end{algorithm}

\subsection{Training Loop}
\begin{algorithm}[htbp]
  \caption{Gradient Descent Training}\label{alg:training}
  \begin{algorithmic}[1]
    \Require Dataset $\mathcal{D}$, learning rate $\eta$, epochs $T$
    \Ensure Trained parameters $\boldsymbol{\theta}^{*}$
    \State Initialise $\boldsymbol{\theta}$ randomly
    \For{$t = 1$ \textbf{to} $T$}
      \For{each mini-batch $\mathcal{B} \subset \mathcal{D}$}
        \State $\mathcal{L} \gets \textsc{ComputeLoss}(\boldsymbol{\theta}, \mathcal{B})$
        \State $\boldsymbol{\theta} \gets \boldsymbol{\theta} - \eta \nabla_{\boldsymbol{\theta}} \mathcal{L}$
      \EndFor
    \EndFor
    \State \Return $\boldsymbol{\theta}$
  \end{algorithmic}
\end{algorithm}

Cross-referencing: \Cref{alg:binary_search} runs in $O(\log n)$;
\Cref{alg:training} follows the standard procedure~\cite{goodfellow2016deep}.

% ============================================================
%  Bibliography
% ============================================================
\bibliographystyle{plainnat}
\bibliography{references}

\end{document}
