% ---------- Tables ----------
\section{Tables}\label{sec:tables}

% --- Simple table ---
\subsection{Simple Table}

\begin{table}[htbp]
  \centering
  \caption{A simple three-column table using \texttt{booktabs}.}
  \label{tab:simple}
  \begin{tabular}{lcc}
    \toprule
    Method   & Accuracy (\%) & Runtime (s) \\
    \midrule
    Baseline & 72.3          & 1.20        \\
    Model A  & 85.1          & 2.45        \\
    Model B  & 88.7          & 3.10        \\
    Model C  & \textbf{91.2} & 4.05        \\
    \bottomrule
  \end{tabular}
\end{table}

% --- Table with multirow and multicolumn ---
\subsection{Table with Multirow and Multicolumn}

\begin{table}[htbp]
  \centering
  \caption{Results grouped by dataset using \texttt{multirow} and
    \texttt{multicolumn}.}
  \label{tab:multi}
  \begin{tabular}{ll cc}
    \toprule
    & & \multicolumn{2}{c}{\textbf{Metrics}} \\
    \cmidrule(lr){3-4}
    \textbf{Dataset} & \textbf{Method} & Precision & Recall \\
    \midrule
    \multirow{2}{*}{CIFAR-10}
      & ResNet-18  & 0.92 & 0.90 \\
      & VGG-16     & 0.89 & 0.87 \\
    \midrule
    \multirow{2}{*}{ImageNet}
      & ResNet-50  & 0.81 & 0.79 \\
      & EfficientNet & 0.84 & 0.82 \\
    \bottomrule
  \end{tabular}
\end{table}

% --- Wide table spanning full text width ---
\subsection{Wide Table}

\begin{table}[htbp]
  \centering
  \caption{A wider table with additional columns.}
  \label{tab:wide}
  \begin{tabular}{l *{5}{c}}
    \toprule
    Method & Param (M) & FLOPs (G) & Top-1 (\%) & Top-5 (\%) & Latency (ms) \\
    \midrule
    Model A & 11.7  & 1.8  & 76.1 & 92.9 & 4.2  \\
    Model B & 23.5  & 3.9  & 78.5 & 94.2 & 7.1  \\
    Model C & 44.6  & 7.8  & 80.3 & 95.1 & 12.8 \\
    Model D & 86.2  & 15.3 & \textbf{82.0} & \textbf{96.0} & 23.4 \\
    \bottomrule
  \end{tabular}
\end{table}

Cross-referencing: \Cref{tab:simple} summarises the baseline comparison;
\Cref{tab:multi} groups results by dataset.
