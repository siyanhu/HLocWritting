\documentclass[12pt, a4paper, onecolumn]{article}

% ---------- Packages ----------
% Layout & formatting
\usepackage[margin=1in]{geometry}
\usepackage{setspace}
\onehalfspacing
\usepackage{parskip}

% Math
\usepackage{amsmath, amssymb, amsthm}

% Graphics
\usepackage{graphicx}
\usepackage{subcaption}
\usepackage{float}

% Tables
\usepackage{booktabs}
\usepackage{multirow}
\usepackage{array}

% Algorithms / pseudocode
\usepackage{algorithm}
\usepackage{algpseudocode}

% References & links
\usepackage[hidelinks]{hyperref}
\usepackage{cleveref}
\usepackage[numbers]{natbib}

% Misc
\usepackage{lipsum} % placeholder text, remove in final version

% ---------- Theorem environments ----------
\newtheorem{theorem}{Theorem}[section]
\newtheorem{lemma}[theorem]{Lemma}
\newtheorem{definition}[theorem]{Definition}

% ---------- Title ----------
\title{Report Title}
\author{
  Author Name \\
  \textit{Affiliation} \\
  \texttt{email@example.com}
}
\date{\today}

% ==========================================================
\begin{document}

\maketitle

\begin{abstract}
  This document serves as a template for academic reports prepared with
  \LaTeX{} and compiled on Overleaf. It demonstrates equations,
  single and multiple figures, tables, and pseudocode blocks in a
  single-column layout.
\end{abstract}

\tableofcontents
\newpage

% ---------- Section files ----------
% ---------- Equations ----------
\section{Equations}\label{sec:equations}

% --- Inline equation ---
An inline equation example: the energy--mass relation is $E = mc^{2}$.

% --- Single numbered equation ---
A standalone numbered equation:
\begin{equation}\label{eq:quadratic}
  x = \frac{-b \pm \sqrt{b^{2} - 4ac}}{2a}
\end{equation}

% --- Multi-line aligned equations ---
A system of aligned equations using \texttt{align}:
\begin{align}
  \nabla \cdot  \mathbf{E} &= \frac{\rho}{\varepsilon_0}       \label{eq:gauss} \\
  \nabla \cdot  \mathbf{B} &= 0                                  \label{eq:gauss_mag} \\
  \nabla \times \mathbf{E} &= -\frac{\partial \mathbf{B}}{\partial t} \label{eq:faraday} \\
  \nabla \times \mathbf{B} &= \mu_0 \mathbf{J}
    + \mu_0 \varepsilon_0 \frac{\partial \mathbf{E}}{\partial t} \label{eq:ampere}
\end{align}

% --- Equation with cases ---
A piecewise function:
\begin{equation}\label{eq:piecewise}
  f(x) =
  \begin{cases}
    x^{2},  & \text{if } x \geq 0, \\
    -x^{2}, & \text{if } x < 0.
  \end{cases}
\end{equation}

% --- Matrix ---
A matrix expression:
\begin{equation}\label{eq:matrix}
  \mathbf{A} =
  \begin{bmatrix}
    a_{11} & a_{12} & a_{13} \\
    a_{21} & a_{22} & a_{23} \\
    a_{31} & a_{32} & a_{33}
  \end{bmatrix}
\end{equation}

% --- Theorem / Definition ---
\begin{definition}[Convexity]\label{def:convex}
  A function $f: \mathbb{R}^n \to \mathbb{R}$ is \emph{convex} if for all
  $\mathbf{x}, \mathbf{y} \in \mathbb{R}^n$ and $\lambda \in [0,1]$:
  \begin{equation}
    f\bigl(\lambda \mathbf{x} + (1-\lambda)\mathbf{y}\bigr)
    \leq \lambda f(\mathbf{x}) + (1-\lambda) f(\mathbf{y}).
  \end{equation}
\end{definition}

Cross-referencing: \Cref{eq:quadratic} shows the quadratic formula;
\Cref{eq:gauss,eq:faraday} are Maxwell's equations.

% ---------- Figures ----------
\section{Figures}\label{sec:figures}

% --- Single figure ---
\subsection{Single Figure}

\begin{figure}[htbp]
  \centering
  % Replace 'example-image' with your own file, e.g. figures/my_plot.pdf
  \includegraphics[width=0.7\textwidth]{example-image}
  \caption{A single figure occupying 70\% of the text width.}
  \label{fig:single}
\end{figure}

% --- Two figures side by side ---
\subsection{Two Figures Side by Side}

\begin{figure}[htbp]
  \centering
  \begin{subfigure}[b]{0.48\textwidth}
    \centering
    \includegraphics[width=\textwidth]{example-image-a}
    \caption{First subfigure.}
    \label{fig:sub_a}
  \end{subfigure}
  \hfill
  \begin{subfigure}[b]{0.48\textwidth}
    \centering
    \includegraphics[width=\textwidth]{example-image-b}
    \caption{Second subfigure.}
    \label{fig:sub_b}
  \end{subfigure}
  \caption{Two subfigures displayed side by side.}
  \label{fig:two_side}
\end{figure}

% --- 2x2 grid of figures ---
\subsection{2$\times$2 Grid of Figures}

\begin{figure}[htbp]
  \centering
  \begin{subfigure}[b]{0.48\textwidth}
    \centering
    \includegraphics[width=\textwidth]{example-image-a}
    \caption{Top left.}
    \label{fig:grid_tl}
  \end{subfigure}
  \hfill
  \begin{subfigure}[b]{0.48\textwidth}
    \centering
    \includegraphics[width=\textwidth]{example-image-b}
    \caption{Top right.}
    \label{fig:grid_tr}
  \end{subfigure}

  \vspace{0.5em}

  \begin{subfigure}[b]{0.48\textwidth}
    \centering
    \includegraphics[width=\textwidth]{example-image-a}
    \caption{Bottom left.}
    \label{fig:grid_bl}
  \end{subfigure}
  \hfill
  \begin{subfigure}[b]{0.48\textwidth}
    \centering
    \includegraphics[width=\textwidth]{example-image-b}
    \caption{Bottom right.}
    \label{fig:grid_br}
  \end{subfigure}
  \caption{A 2$\times$2 grid of subfigures sharing one caption.}
  \label{fig:grid}
\end{figure}

% --- Three figures in a row ---
\subsection{Three Figures in a Row}

\begin{figure}[htbp]
  \centering
  \begin{subfigure}[b]{0.31\textwidth}
    \centering
    \includegraphics[width=\textwidth]{example-image-a}
    \caption{First.}
    \label{fig:tri_a}
  \end{subfigure}
  \hfill
  \begin{subfigure}[b]{0.31\textwidth}
    \centering
    \includegraphics[width=\textwidth]{example-image-b}
    \caption{Second.}
    \label{fig:tri_b}
  \end{subfigure}
  \hfill
  \begin{subfigure}[b]{0.31\textwidth}
    \centering
    \includegraphics[width=\textwidth]{example-image-c}
    \caption{Third.}
    \label{fig:tri_c}
  \end{subfigure}
  \caption{Three subfigures arranged in a single row.}
  \label{fig:triple}
\end{figure}

Cross-referencing: see \Cref{fig:single} for the single figure and
\Cref{fig:sub_a,fig:sub_b} for the side-by-side pair.

% ---------- Tables ----------
\section{Tables}\label{sec:tables}

% --- Simple table ---
\subsection{Simple Table}

\begin{table}[htbp]
  \centering
  \caption{A simple three-column table using \texttt{booktabs}.}
  \label{tab:simple}
  \begin{tabular}{lcc}
    \toprule
    Method   & Accuracy (\%) & Runtime (s) \\
    \midrule
    Baseline & 72.3          & 1.20        \\
    Model A  & 85.1          & 2.45        \\
    Model B  & 88.7          & 3.10        \\
    Model C  & \textbf{91.2} & 4.05        \\
    \bottomrule
  \end{tabular}
\end{table}

% --- Table with multirow and multicolumn ---
\subsection{Table with Multirow and Multicolumn}

\begin{table}[htbp]
  \centering
  \caption{Results grouped by dataset using \texttt{multirow} and
    \texttt{multicolumn}.}
  \label{tab:multi}
  \begin{tabular}{ll cc}
    \toprule
    & & \multicolumn{2}{c}{\textbf{Metrics}} \\
    \cmidrule(lr){3-4}
    \textbf{Dataset} & \textbf{Method} & Precision & Recall \\
    \midrule
    \multirow{2}{*}{CIFAR-10}
      & ResNet-18  & 0.92 & 0.90 \\
      & VGG-16     & 0.89 & 0.87 \\
    \midrule
    \multirow{2}{*}{ImageNet}
      & ResNet-50  & 0.81 & 0.79 \\
      & EfficientNet & 0.84 & 0.82 \\
    \bottomrule
  \end{tabular}
\end{table}

% --- Wide table spanning full text width ---
\subsection{Wide Table}

\begin{table}[htbp]
  \centering
  \caption{A wider table with additional columns.}
  \label{tab:wide}
  \begin{tabular}{l *{5}{c}}
    \toprule
    Method & Param (M) & FLOPs (G) & Top-1 (\%) & Top-5 (\%) & Latency (ms) \\
    \midrule
    Model A & 11.7  & 1.8  & 76.1 & 92.9 & 4.2  \\
    Model B & 23.5  & 3.9  & 78.5 & 94.2 & 7.1  \\
    Model C & 44.6  & 7.8  & 80.3 & 95.1 & 12.8 \\
    Model D & 86.2  & 15.3 & \textbf{82.0} & \textbf{96.0} & 23.4 \\
    \bottomrule
  \end{tabular}
\end{table}

Cross-referencing: \Cref{tab:simple} summarises the baseline comparison;
\Cref{tab:multi} groups results by dataset.

% ---------- Algorithms / Pseudocode ----------
\section{Algorithms}\label{sec:algorithms}

% --- Simple algorithm ---
\subsection{Simple Algorithm}

\begin{algorithm}[htbp]
  \caption{Binary Search}\label{alg:binary_search}
  \begin{algorithmic}[1]
    \Require Sorted array $A[1 \ldots n]$, target value $t$
    \Ensure Index $i$ such that $A[i] = t$, or $-1$ if not found
    \State $\ell \gets 1$
    \State $r \gets n$
    \While{$\ell \leq r$}
      \State $m \gets \lfloor (\ell + r) / 2 \rfloor$
      \If{$A[m] = t$}
        \State \Return $m$
      \ElsIf{$A[m] < t$}
        \State $\ell \gets m + 1$
      \Else
        \State $r \gets m - 1$
      \EndIf
    \EndWhile
    \State \Return $-1$
  \end{algorithmic}
\end{algorithm}

% --- Algorithm with functions and loops ---
\subsection{Training Loop}

\begin{algorithm}[htbp]
  \caption{Gradient Descent Training}\label{alg:training}
  \begin{algorithmic}[1]
    \Require Dataset $\mathcal{D}$, learning rate $\eta$, epochs $T$
    \Ensure Trained parameters $\boldsymbol{\theta}^{*}$
    \State Initialise $\boldsymbol{\theta}$ randomly
    \For{$t = 1$ \textbf{to} $T$}
      \For{each mini-batch $\mathcal{B} \subset \mathcal{D}$}
        \State $\mathcal{L} \gets \textsc{ComputeLoss}(\boldsymbol{\theta}, \mathcal{B})$
        \State $\mathbf{g} \gets \nabla_{\boldsymbol{\theta}} \mathcal{L}$
        \State $\boldsymbol{\theta} \gets \boldsymbol{\theta} - \eta \, \mathbf{g}$
      \EndFor
      \If{convergence criterion met}
        \State \textbf{break}
      \EndIf
    \EndFor
    \State $\boldsymbol{\theta}^{*} \gets \boldsymbol{\theta}$
    \State \Return $\boldsymbol{\theta}^{*}$
  \end{algorithmic}
\end{algorithm}

Cross-referencing: \Cref{alg:binary_search} runs in $O(\log n)$ time;
\Cref{alg:training} outlines the standard training procedure.


% ---------- Bibliography ----------
\bibliographystyle{plainnat}
\bibliography{references}

\end{document}
